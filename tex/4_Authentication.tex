\chapter{Authentication \& Authorization}

The BI-DBS portal transfers to microservices architecture and getting\\  
 modernized. In order to keep the application secure we need to design a new identity and access management. In this chapter, I will describe the design for implementation of Authorization and Authorization including the user roles and permissions management.

\section{Authentication}\label{sec:authentication}

Authentication is a process of identification and verification of user's access to the application. Authentication microservice was designed and described along with general communication flow for this service by Ing. Andrii Plyskach in his master thesis. I am going to describe and implement authozation for a client side using endpoints of that microservice.
\\Authorization microservice is designed using OAuth protocol. This is a modern industry-standard protocol for authorization. The concept of that protocol is to grand an access to a user without a need for a user to share their credentials with an application they want to access. 
\\ Unauthorized user trying to access the application is always redirected to a log in page where user is supposed to provide an authorization server with credentials aby filling in a log in form with their credentials and submitting the form. If authorization by authoriation server was succesfull, authorization server will do redirect to a url with a code which can be later exchanged for authorization token. 


\subsection{Access token}
what is this

\subsection{Refresh token}

\subsection{Tokens security}
how we keep them secure
compare different approaches for storing tokens

\subsection{Restrict access for non-authorized users}
router... and method ...


\section{Authorization}\label{sec:authorization}

\subsection{Roles}

\subsection{Permissions}






\section{DevOps concepts and practices} DevOps concepts are a common set of rules which are the core of this methodology. It is not just a set of tools, but a cultural philosophy, a way of project life-cycle organization. ... There are no strictly defined concepts... I have combined the philosophy, and described concepts to 5 base concepts for adopting DevOps on existing project..

\subsection{Data-Based Decision Making} With the DevOps approach decisions from choosing a technology stack for the application to adding features should be made based on collected data. The first part should be always collecting as much relevant data as it is possible. Then, based on the collected data analysis of the team, a decision should be made. It helps to create software that solves real problems effectively. Decisions made without considering client feedback data, colleagues' opinions, and proper analysis would lead to creating badly-functional software full of useless features which does not fulfill the client's needs.

\subsection{Collaboration} The term DevOps itself is a collaboration of two words as well as one of its concepts is a collaboration of different IT departments. That means that the roles in the team are not so strict and independent as in a traditional work and team organization. Developer and IT operations roles are getting closer to full-stack roles. Leading to a better understanding of the software development life-cycle by the whole team.\\
It goes great together with the microservices architecture which is, by the way, becoming a standard for DevOps model. A developer does not have the main focus on understanding the logic of the application because it's simplified by the architecture but instead is more focused on technological and operational processes. 

\subsection{Responsibility Throughout the Lifecycle} DevOps methodology comes with a requirement for team members to fully understand the process of software development from the feature idea to implementation and deployment and take responsibility for it.\\
End-to-end responsibility helps to reduce failures and resolve bugs quickly.
% When all team members are aware of the 

\subsection{Constant Improvement} Constant improvement is a special concept and practice of DevOps methodology. The main idea is a focus on improvements, updates, and experimenting. It tells each team member not to be afraid of failures but take them as an opportunity to learn, whatever the outcome of an experiment, a person will have a deeper understanding of what works and what does not. Besides this rule gives more responsibility to a person and allows them to consistently push code changes to minimize waste, do speed optimization and improve development efficiency.

\subsection{Automation} DevOps approach is meant to benefit in fast development and improvement. Needless to mention that automation is one of the golden rules for increasing the speed of the application life-cycle. Everyone should aim to automate as many phases of the process as it is possible. As a result team members are satisfied with a decreased need for doing repetitive tasks. Thus they can focus on significant tasks and work on new features. Overall it helps minimize human errors and boosts team output. \\

\noindent Together with the constant improvement concept, automation is represented by a set of practices.



\paragraph*{Continuous integration (CI)}

\paragraph*{Continuous deployment (CD)}

\paragraph*{Continuous delivery (CD)}

\paragraph*{Continuous monitoring (CM)}

\paragraph*{Continuous testing}

\paragraph*{Infrastructure as Code (IaC)}


% Automation is a key element of a CI/CD pipeline, more about CI/CD in section 2.3.








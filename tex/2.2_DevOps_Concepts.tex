\section{DevOps concepts} DevOps concepts are a common set of rules which are the core of this methodology. It is not just a set of tools, but a cultural philosophy, a way of project lifecycle organization. Those rules are not strictly defined, they come from a DevOps culture and can be interpreted differently describing the same model. In this section, I will analyze and combine the culture philosophy and most frequently mentioned rules\cite{devops-conc1, devops-conc2} in five concepts that represent the DevOps methodology.

\subsection{Automation} DevOps approach is meant to benefit in fast development and improvement. Needless to mention that automation is one of the golden rules for increasing the speed of the application lifecycle. Everyone in a team should aim to automate as many phases of the process as it is possible. As a result, team members are satisfied with a decreased need for doing repetitive tasks. Thus they can focus on significant tasks and work on new features. Overall it helps minimize human errors and boosts team output.

\paragraph*{Usage in the BI-DBS project} This concept was the main reason for me to consider adopting DevOps model in the BI-DBS project. Due to microservices architecture the project needs to have new automated deployment and testing processes. These and other automation practices we might want to adopt are described in subsection 2.3.


\subsection{Data-Based Decision Making} With the DevOps approach, decisions from choosing a technology stack for the application to adding features should be made based on collected data. The first part of making a decision should always be collecting as much relevant data as it is possible. Then, based on the collected data analysis of the team, a decision should be made. It helps to create software that solves real problems effectively. Decisions made without considering client feedback data, colleagues' opinions, and proper analysis would lead to creating badly-functional software full of useless features which does not fulfill the client's needs.

\paragraph*{Usage in the BI-DBS project} This concept is very suitable for our new growing project since in the current phase we create are creating a core which should be done properly based on analyses of collected data to avoid having useless features, too complex design and irrelevant technologies.

\subsection{Responsibility Throughout the Lifecycle} DevOps methodology comes with a requirement for team members to fully understand the process of software development from the feature idea to implementation and deployment and take responsibility for it. End-to-end responsibility helps to reduce failures and resolve bugs quickly.

\paragraph*{Usage in the BI-DBS project} From my experience, students usually want to finish their part of the job as fast as it is possible. Therefore they sometimes tend to skip spending time to understand the idea of the task properly. Thus they get to implement it without thinking of the consequences their changes might cause. Moreover, they do not always get to test it properly. It is essential to integrate this concept more into development student teams to increase the quality of produced code.

\subsection{Constant Improvement} Constant improvement is a special concept and practice of DevOps methodology. The main idea is a focus on improvements, updates, and experimenting. It tells each team member not to be afraid of failures but take them as an opportunity to learn. Whatever the outcome of an experiment, a person will have a deeper understanding of what works and what does not. Besides, this rule gives more responsibility to a person and allows them to consistently push code changes to minimize waste, do speed optimization and improve development efficiency.

\paragraph*{Usage in the BI-DBS project} This concept is friendly for students in a way that they can try new things without fear of failure if things do not work out. Adopting this concept will benefit the project in case it is used with the two previous concepts, otherwise, students might take the development less seriously and do experiments only for the purpose of faster finishing the task, but not improving.

\subsection{Collaboration} This concept is a collaboration of different IT departments on the project. That means that the team's roles are not as strict and independent as in a traditional work and team organization. Developer and operation roles are getting closer to full-stack roles, leading to a better understanding of the software development lifecycle by the whole team.

\paragraph*{Usage in the BI-DBS project} This illustrating DevOps methodology concept is beneficial in two ways for the BI-DBS portal. Firstly, students will learn essential operation processes and understand the basics of automation. Secondly, the portal always needs at least one person to be available to manage application operations. Using this concept will increase the number of people who understand the processes and thus are able to manage operations in case of a need.








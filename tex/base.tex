% arara: xelatex
% arara: xelatex
% arara: xelatex


% options:
% thesis=B bachelor's thesis
% thesis=M master's thesis
% czech thesis in Czech language
% english thesis in English language
% hidelinks remove colour boxes around hyperlinks

\documentclass[thesis=B,english]{template/FITthesis}[2019/12/23]

%\usepackage[utf8]{inputenc} % LaTeX source encoded as UTF-8
% \usepackage[latin2]{inputenc} % LaTeX source encoded as ISO-8859-2
% \usepackage[cp1250]{inputenc} % LaTeX source encoded as Windows-1250

% \usepackage{subfig} %subfigures
% \usepackage{amsmath} %advanced maths
% \usepackage{amssymb} %additional math symbols

\usepackage{dirtree} %directory tree visualisation
\usepackage{biblatex}


% \bibliographystyle{iso690}

\bibliography{tex/mybibliographyfile}


% % list of acronyms
% \usepackage[acronym,nonumberlist,toc,numberedsection=autolabel]{glossaries}
% \iflanguage{czech}{\renewcommand*{\acronymname}{Seznam pou{\v z}it{\' y}ch zkratek}}{}
% \makeglossaries


\department{Department of Software Engineering}
\title{DevOps concepts - CI/CD, implementation of authorization \& authentication, presented on a BI-DBS portal frontend}
\authorGN{Volha} %author's given name/names
\authorFN{Chukava} %author's surname
\author{Volha Chukava} %author's name without academic degrees
\authorWithDegrees{Volha Chukava} %author's name with academic degrees
\supervisor{Ing. Oldřich Malec}
\placeForDeclarationOfAuthenticity{Prague}

\acknowledgements{I would like to thank my supervisor Ing. Oldřich Malec for constructive and valuable feedback, giving me the flexibility in constructing my thesis and assisstance in writeing the thesis. 

I would like to thank a person who made it possible for me to write this thesis Ing. Jiri Hunka, who has inspired me in the begining 

I would like to acknowledge the contribution of Ing. Adnrii Plyskach and Bc. Max Hejda, who have shared their knowledge, expertise, and resources, and have played an important role in helping me to achieve my thesis goals.

My friend Bc. Darya Litvinenka and my partner Bc. Adam Vonasek for their support and time spent testing my application}
\abstractEN{A new application is being developed for teaching the BI-DBS subject at FIT CTU in Prague. The requirements for the application are derived from the existing application. This bachelor's thesis is focused on improving the development and maintenance processes of the new frontend for the web application based on analyses of the state of the existing application and the planned state. The main improvement points are adopting DevOps methodology and designing a new simplified and clear access management system. Conclusively, the main goal of the thesis is the implementation of authentication and authorization including restricting permissions, and automation of testing and deployment.

}
\abstractCS{Pro předmět BI-DBS na FIT ČVUT v Praze je vyvíjena nová aplikace. Požadavky pro tuto aplikaci jsou odvozeny z již existující aplikace. Tato bakalářská práce se zaměřuje na zlepšení procesů vývoje a údržby nového frontendu webové aplikace na základě analýzy stavu stávající aplikace a plánovaného stavu. Hlavními oblastmi zlepšení jsou přijetí metodiky DevOps a navržení nového zjednodušeného a přehledného systému řízení přístupů. Nakonec hlavním cílem této práce je implementace autentizace a autorizace, včetně omezení oprávnění, a automatizace procesů testování a nasazování.}
\keywordsCS{webová aplikace, frontend, CI, CD, OAuth, autentizace, autorizace

\newpage 

\ % The empty page

\newpage}
\keywordsEN{Replace with comma-separated list of keywords in English.}

% TODO: choose a right one
\declarationOfAuthenticityOption{1} %select as appropriate, according to the desired license (integer 1-6)
% \website{http://site.example/thesis} %optional thesis URL


\begin{document}

% \newacronym{CVUT}{{\v C}VUT}{{\v C}esk{\' e} vysok{\' e} u{\v c}en{\' i} technick{\' e} v Praze}
% \newacronym{FIT}{FIT}{Fakulta informa{\v c}n{\' i}ch technologi{\' i}}


\setsecnumdepth{part}
\chapter{Introduction}

TODO: zminit navazani na bakalarky a diplomky pouzite pro psani teto prace

\setsecnumdepth{all}

\chapter{Analysis of the application state}
In this chapter, I will describe the current and planned state of the BI-DBS portal from the perspective of application architecture, design patterns and used technology stack. The goal is to identify the existing problems of the current application. Outline how some of them will be solved in a new portal, as well as indicate what difficulties we can face developing the BI-DBS portal using a new stack of technologies and new architecture.


\section{Current state of the application}
The current BI-DBS portal was first designed and implemented in 2016. Over time it gained new features and grew large. Used technologies became not relevant and it became difficult to maintain. 

\subsection{Architecture}
The current application was built in a traditional way, using a monolithic architecture approach and following the Model-View-Presenter pattern\cite{potel_mvp}. That means that the whole application is presented as one monolithic unit, and it is composed  of three components.

\begin{itemize}
  \item The model: Communicates with the database and handles domain logic
  \item The view: Provides visualization and directs user commands to the presenter
  \item The presenter: Manages interactions between the database and the view
\end{itemize}

\noindent
This architecture's main concept is having one code base that benefits in simplifying development, testing, debugging, and deployment. \\
However, we can have those benefits only until the application grows large. Then all those processes get slower, more complex, and become problematic. In addition, with a lack of flexibility and scalability, it becomes challenging to maintain the application and keep it secure. \\

\noindent The BI-DBS portal is being developed by students. Students generally do not have much experience developing large applications and dealing with complex dependencies. Besides, they have limited time to progress in learning and developing the portal. Therefore it takes a lot of time for students to learn before contributing to the project. Thus it is more challenging to keep the application maintainable and ... (just one more example)


\subsection{Technologies}
\paragraph*{PHP.} PHP is a general-purpose, open-source scripting language that can be integrated into HTML. It differs from client-side scripting languages in that its HTML is generated on a server and then sent to a client. That feature allows to rapidly build a web application with a thick-server and thin-client. This is a one of the approaches of how to use PHP to build an application and it is used in a current project.
\paragraph*{Nette.} This framework is a core of the current applciation as it manages 

\paragraph*{Latte.} 

\paragraph*{Webpack.}




% \subsection{Sources}


% HTML - https://developer.mozilla.org/en-US/docs/Web/HTML
% php - https://www.php.net/manual/en/intro-whatis.php
% monolithic - https://www.atlassian.com/microservices/microservices-architecture/microservices-vs-monolith ,
% https://www.qulix.com/about/monolithic-vs-microservices-architechture/



% \section{Planned state of the application}
% \subsection{Architecture}
% \subsection{Technologies}
 
% \chapter{DevOps Concepts, CI/CD.}

\section{What is DevOps?}

\section{DevOps concepts}

\section{Continious Integration}

\section{Continious Delivery}
% \input{tex/3_Deployment}
% \chapter{Authentication \& Authorization}

The BI-DBS portal transfers to microservices architecture and gets \\  
 modernized. In order to keep the application secure, we need to design a new identity and access management. In this chapter, I will describe the design for implementation of Authorization and Authorization, including the user roles and permissions management.

\section{Authentication}\label{sec:authentication}

Authentication is a process of identification and verification of the user's access to the application. The authentication microservice was designed and described along with the general communication flow for this service by Ing. Andrii Plyskach in his master thesis. I will describe and implement authorization for a client-side using endpoints of that microservice.


\subsection{OAuth protocol}
\\Authorization microservice is designed using OAuth protocol. This is a modern industry-standard protocol for authorization. The concept of that protocol is to grant access to a user without a need for a user to share their credentials with an application they want to access.

 
 \subsection{Authentication flow}
 An unauthenticated user trying to access the application is always redirected to a login page where a user will provide an authorization server with credentials by filling in a login form and submitting it. If authorization by the authorization server were successful, the authorization server will do redirect to a URL with a code. From that moment frontend does not communicate with an authorization service untill refresh token is expired.



Example: HTTP......

that can be later exchanged for an authorization token. 





description of what data we get from the BE
% {
% access_token,
% refresh_token
% }

\subsection{Access token}
The access token is a short-time living token that is used to verify the user's access to the application on both frontend and backend sides. An access token is sent in every request sent to backend, so backend can verify if the token is valid and only then complete a request.

\subsection{Refresh token}
The refresh token is meant to be a long-time living token. Because it is supposed to have a longer expiration time to be able to refresh an access token.
However, our microservice supports refresh token rotation, which means that every time we refresh the access token we also get a new refresh token, which makes its time living much shorter and thus more secure. There are different opinions about what is the best and most secure way of storing this token. [OAuth] pages say that with use of token rotation using local storage is possible

\subsection{Tokens security}
how do we keep them secure
compare different approaches for storing tokens


\section{Authorization}\label{sec:authorization}

\subsection{Roles}

\subsection{Permissions}






% \input{tex/5_Authorization}
\input{tex/6_Tests}
\chapter{Realisation}  % add impl separetely


\setsecnumdepth{part}
\chapter{Conclusion}



\nocite{*}
\printbibliography[title={vfvf}]


\setsecnumdepth{all}
% \appendix

\chapter{Acronyms}
% \printglossaries
\begin{description}
   \item[MVP] Model View Presenter
   \item[PHP] Hypertext Preprocessor
   \item[HTML] HyperText Markup Language
\end{description}


\chapter{Contents of enclosed CD}



%change appropriately

% \begin{figure}
% 	\dirtree{%
% 		.1 readme.txt\DTcomment{the file with CD contents description}.
% 		.1 exe\DTcomment{the directory with executables}.
% 		.1 src\DTcomment{the directory of source codes}.
% 		.2 wbdcm\DTcomment{implementation sources}.
% 		.2 thesis\DTcomment{the directory of \LaTeX{} source codes of the thesis}.
% 		.1 text\DTcomment{the thesis text directory}.
% 		.2 thesis.pdf\DTcomment{the thesis text in PDF format}.
% 		.2 thesis.ps\DTcomment{the thesis text in PS format}.
% 	}
% \end{figure}






% Cíle práce

% \chapter{Analýza a návrh}

% % \begin{abstract}
% %         V této kapitole provedeme analýzu požadavků a návrh včetně zdůvodnění všech rozhodnutí.
% % \end{abstract}

% \section{Analýza}

% % ``smth'' 
% % --
% % -

% Přidáme odstavec Text - zejména ten odborný - je nutné členit na odstavce. Každý odstavec by se měl týkat jednoho tématu, myšlenky... Odstavce od~ sebe~ musí být vizuálně oddělené. K tomu existuje několik vhodných stylů, které si~ popíšeme v jedné z následujících kapitol. Odstavce mohou být různě vysázené. V odborných textech je běžná sazba "do bloku". Při ní je nutné vhodně měnit mezislovní mezery. Jejich doporučená velikost je 0,25 - 0.33 čtverčíku.

% Požadavky jsou těchto typů:
% \begin{description}
%     \item[funkční:] objasňují, co se musí udělat a identifikují nutné aktivity;
%     \item[nefunkční:] jsou všechny, které nejsou funkční. Typicky mezi ně patří: 
%         \begin{itemize}
%             \item výkonnostní
%             \item designové
%         \end{itemize}
% \end{description}


% \subsection{Funkční požadavky}

% Funkční požadavky této práce jsou:
% \begin{enumerate}
%     \item získat zadání
%     \item sepsat práci
%     \item včas odevzdat
%     \item obhájit \footnote{Když se zadaří.}
% \end{enumerate}


% \section{Tabulky}

% V tabulce 3.1 najdete možnosti, jak získat body v předmětu BI-DPR.


% \begin{tabular}{ p{3cm}|c|c }
%      činnost & body & povinná\\
%      Zpracování a odevzdání zápočtového projektu (rozpracované bakalářské práce) & 30 & ANO\\
%      test z typografie & 10 & ANO\\
%      test z citací & 10 & ANO\\
%      Udělení bodů od vedoucího bakalářské práce & 20 & NE
% \end{tabular}



% Logo. Získáno z \url{https://www.cvut.cz/logo-a-graficky-manual}

\end{document}




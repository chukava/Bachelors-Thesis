% arara: xelatex
% arara: xelatex
% arara: xelatex


% options:
% thesis=B bachelor's thesis
% thesis=M master's thesis
% czech thesis in Czech language
% english thesis in English language
% hidelinks remove colour boxes around hyperlinks

\documentclass[thesis=B,english]{template/FITthesis}[2019/12/23]

%\usepackage[utf8]{inputenc} % LaTeX source encoded as UTF-8
% \usepackage[latin2]{inputenc} % LaTeX source encoded as ISO-8859-2
% \usepackage[cp1250]{inputenc} % LaTeX source encoded as Windows-1250

% \usepackage{subfig} %subfigures
% \usepackage{amsmath} %advanced maths
% \usepackage{amssymb} %additional math symbols

\usepackage{dirtree} %directory tree visualisation
\usepackage{biblatex}


% \bibliographystyle{iso690}

\bibliography{tex/mybibliographyfile}


% % list of acronyms
% \usepackage[acronym,nonumberlist,toc,numberedsection=autolabel]{glossaries}
% \iflanguage{czech}{\renewcommand*{\acronymname}{Seznam pou{\v z}it{\' y}ch zkratek}}{}
% \makeglossaries


\department{Department of Software Engineering}
\title{DevOps concepts - CI/CD, implementation of authorization \& authentication, presented on a BI-DBS portal frontend}
\authorGN{Volha} %author's given name/names
\authorFN{Chukava} %author's surname
\author{Volha Chukava} %author's name without academic degrees
\authorWithDegrees{Volha Chukava} %author's name with academic degrees
\supervisor{Ing. Oldřich Malec}
\placeForDeclarationOfAuthenticity{Prague}

\acknowledgements{I would like to thank ...}
\abstractEN{A new application is being developed for teaching the BI-DBS subject at FIT CTU in Prague. The requirements for the application are derived from the existing application. This bachelor's thesis is focused on improving the development and maintenance processes of the new frontend for the web application based on analyses of the state of the existing application and the planned state. The main improvement points are adopting DevOps methodology and designing a new simplified and clear access management system. Conclusively, the main goal of the thesis is the implementation of authentication and authorization including restricting permissions, and automation of testing and deployment.

}
\abstractCS{V n{\v e}kolika v{\v e}t{\' a}ch shr{\v n}te obsah a p{\v r}{\' i}nos t{\' e}to pr{\' a}ce v {\v c}esk{\' e}m jazyce.}
\keywordsCS{webová aplikace, frontend, CI, CD, OAuth, autentizace, autorizace

\newpage 

\ % The empty page

\newpage}
\keywordsEN{web application, DevOps, CI, CD, OAuth, frontend, authentication, authorizarion}

% TODO: choose a right one
\declarationOfAuthenticityOption{1} %select as appropriate, according to the desired license (integer 1-6)
% \website{http://site.example/thesis} %optional thesis URL


\begin{document}

% \newacronym{CVUT}{{\v C}VUT}{{\v C}esk{\' e} vysok{\' e} u{\v c}en{\' i} technick{\' e} v Praze}
% \newacronym{FIT}{FIT}{Fakulta informa{\v c}n{\' i}ch technologi{\' i}}


\setsecnumdepth{part}
\chapter{Introduction}

\setsecnumdepth{all}

\chapter{Analysis}

In this chapter I am going to describe  
% \chapter{DevOps Concepts, CI/CD}
% \chapter{Deployment}
% \chapter{Authentication \& Authorization}

The BI-DBS portal transfers to microservices architecture and getting\\  
 modernized. In order to keep the application secure we need to design a new identity and access management. In this chapter, I will describe the design for implementation of Authorization and Authorization including the user roles and permissions management.

\section{Authentication}\label{sec:authentication}

Authentication is a process of identification and verification of user's access to the application. Authentication microservice was designed and described along with general communication flow for this service by Ing. Andrii Plyskach in his master thesis. I am going to describe and implement authozation for a client side using endpoints of that microservice.
\\Authorization microservice is designed using OAuth protocol. This is a modern industry-standard protocol for authorization. The concept of that protocol is to grand an access to a user without a need for a user to share their credentials with an application they want to access. 
\\ Unauthorized user trying to access the application is always redirected to a log in page where user is supposed to provide an authorization server with credentials aby filling in a log in form with their credentials and submitting the form. If authorization by authoriation server was succesfull, authorization server will do redirect to a url with a code which can be later exchanged for authorization token. 


\subsection{Access token}
what is this

\subsection{Refresh token}

\subsection{Tokens security}
how we keep them secure
compare different approaches for storing tokens

\subsection{Restrict access for non-authorized users}
router... and method ...


\section{Authorization}\label{sec:authorization}

\subsection{Roles}

\subsection{Permissions}






% \chapter{Authorization}
\chapter{Tests}
\chapter{Realisation}  % add impl separetely


\setsecnumdepth{part}
\chapter{Conclusion}



\nocite{*}
\printbibliography[title={vfvf}]


\setsecnumdepth{all}
% \appendix

\chapter{Acronyms}
% \printglossaries
\begin{description}
   \item[MVP] Model View Presenter
   \item[PHP] Hypertext Preprocessor
   \item[HTML] HyperText Markup Language
\end{description}


\chapter{Contents of enclosed CD}



%change appropriately

% \begin{figure}
% 	\dirtree{%
% 		.1 readme.txt\DTcomment{the file with CD contents description}.
% 		.1 exe\DTcomment{the directory with executables}.
% 		.1 src\DTcomment{the directory of source codes}.
% 		.2 wbdcm\DTcomment{implementation sources}.
% 		.2 thesis\DTcomment{the directory of \LaTeX{} source codes of the thesis}.
% 		.1 text\DTcomment{the thesis text directory}.
% 		.2 thesis.pdf\DTcomment{the thesis text in PDF format}.
% 		.2 thesis.ps\DTcomment{the thesis text in PS format}.
% 	}
% \end{figure}






% Cíle práce

% \chapter{Analýza a návrh}

% % \begin{abstract}
% %         V této kapitole provedeme analýzu požadavků a návrh včetně zdůvodnění všech rozhodnutí.
% % \end{abstract}

% \section{Analýza}

% % ``smth'' 
% % --
% % -

% Přidáme odstavec Text - zejména ten odborný - je nutné členit na odstavce. Každý odstavec by se měl týkat jednoho tématu, myšlenky... Odstavce od~ sebe~ musí být vizuálně oddělené. K tomu existuje několik vhodných stylů, které si~ popíšeme v jedné z následujících kapitol. Odstavce mohou být různě vysázené. V odborných textech je běžná sazba "do bloku". Při ní je nutné vhodně měnit mezislovní mezery. Jejich doporučená velikost je 0,25 - 0.33 čtverčíku.

% Požadavky jsou těchto typů:
% \begin{description}
%     \item[funkční:] objasňují, co se musí udělat a identifikují nutné aktivity;
%     \item[nefunkční:] jsou všechny, které nejsou funkční. Typicky mezi ně patří: 
%         \begin{itemize}
%             \item výkonnostní
%             \item designové
%         \end{itemize}
% \end{description}


% \subsection{Funkční požadavky}

% Funkční požadavky této práce jsou:
% \begin{enumerate}
%     \item získat zadání
%     \item sepsat práci
%     \item včas odevzdat
%     \item obhájit \footnote{Když se zadaří.}
% \end{enumerate}


% \section{Tabulky}

% V tabulce 3.1 najdete možnosti, jak získat body v předmětu BI-DPR.


% \begin{tabular}{ p{3cm}|c|c }
%      činnost & body & povinná\\
%      Zpracování a odevzdání zápočtového projektu (rozpracované bakalářské práce) & 30 & ANO\\
%      test z typografie & 10 & ANO\\
%      test z citací & 10 & ANO\\
%      Udělení bodů od vedoucího bakalářské práce & 20 & NE
% \end{tabular}



% Logo. Získáno z \url{https://www.cvut.cz/logo-a-graficky-manual}

\end{document}




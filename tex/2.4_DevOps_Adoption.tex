\section{DevOps adoption} The idea of adopting the DevOps model came to me with a need to configure the new deployment of the new BI-DBS portal due to the transfer to microservices architecture. Before analyzing the DevOps concepts, I assumed that the DevOps model is just an automation idea. In fact, I was wrong and did not know it is a solid methodology bringing huge advantages to the project. From my own observations, it is a pretty common misunderstanding of the DevOps model, which leads to missing out important concepts.\\ 

\noindent "Even while automation helps speed up manual operations, cooperation and communication are the key objectives of DevOps. Automating your operations won’t bring about the desired business benefits unless everyone involved in the software development, delivery, testing, and operating processes adopts excellent communication and collaborative practices." \cite{devops-adoption}\\
The analysis makes it clear that the DevOps model is suitable and valuable for the BI-DBS portal project management and development.

\paragraph*{Adoption steps:} 


\begin{enumerate}
    \item \emph{Devops philosophy.} This thesis can be used to introduce the DevOps methodology to students. Before getting to development as well as learning the processes of development students should learn team organization management including DevOps concepts. 
    \item \emph{DevOps Practices.} Analyze which practice we would like to adopt and how it will be beneficial and then complete the three next steps:
        \begin{enumerate}
            \item Choosing relevant tools for a practice we would like to adopt
            \item Application of the practice using chosen tools
            \item Document the configuration of the practice for a team
        \end{enumerate}
    I will adopt the most important DevOps practices for the BI-DBS portal in the automation chapter using these steps.
\end{enumerate}



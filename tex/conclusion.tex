\chapter{Conclusion} First of the thesis goals was contributing to the project by improving the development efficiency and maintenance, which was achieved by introducing the DevOps methodology and also automating such processes as testing and development. The next goal was defined as adjusting the project to make it more structured and secure. Therefore, I have analyzed the current access management structure and came to the conclusion that it is unnecessarily too complicated. I have managed to logically simplify it and provide a design of a new access management system. Moreover, based on the clear permissions system I have strengthened the security by providing validation before every user's request. Finally, I planned to implement the authentication and authorization features. These functionalities were created using the OAuth 2.0 protocol and authorization microservice. Furthermore, I have reduced security vulnerabilities by providing a suitable way of storing sensitive data.\\
In my opinion, this thesis will definitely be useful for developers and managers of the BI-DBS portal. It can be used as a part of introduction materials to DevOps methodology for new students before getting to the application development as well as the documentation of the access management system for managers and developers.\\
From the analysis of the current and planned application state, I have outlined three challenges, which we are facing in the new application state due to the microservices architecture. I have provided the solutions for two of them. The deployment problem was resolved by automation of this process. Possible security vulnerabilities were decreased by creating an access management system. The testing challenge still remains unresolved. The application needs a strong set of integration tests. Ideally, they should be a part of CI/CD pipeline to reduce the probability of deploying errors. This is an idea for further improvement which can become a topic for a thesis or a set of tasks for the teams of BI-DBS developers.\\
An integral part of working on the development of the BI-DBS portal is working in team. I am endlessly grateful for the collaboration with such amazing and skillful people and the experience I have gained. Thanks to this thesis and the BI-DBS project, I have learned a lot about application security and efficient development. Additionally, through the development of various features, I have gained experience working with modern technologies such as Vue.js, TypeScript, Pinia, and others.
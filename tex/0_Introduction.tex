\chapter{Introduction}

This thesis is a continuation of my development journey of the BI-DBS web application, which started in February 2022 as a part of the frontend development team. The project was very complex and challenging for maintenance and development. Not a long time after our team got into analyzing and developing the current application, a new solution to the maintenance problem was introduced to us by Ing. Andrii Plyskach. He came up with a new architectural design based on microservices architecture for the application, which he described in his master thesis\cite{mt-plyskach}. The decision was made to follow that design, which meant creating a new project based on the requirements of the current application. The development process, as well as the project, was divided into client and server side parts.\\
For our team, it meant starting frontend development from the very beginning using the new technologies. Thus it gave us a lot of space for our ideas, but also a big responsibility. Therefore, since the creation of a new frontend project, my goal was to configure the project in a way to make it structured, organize efficient development in a team as well as work on the implementation of the features. This goal remains the same for this thesis.\\
Firstly, I am going to analyze the current application state and the planned state, to see what we can expect from the changes. Secondly, for improving the efficiency of the software development process and application improvement and maintenance I will introduce the DevOps model and the instruction for its adoption. Besides, I will use that instruction for automating testing and development processes for the frontend. Finally, I am going to implement authorization and implementation based on OAuth 2.0 protocol using the authorization microservice.
